\section*{Abstract}

本研究は, 同時通訳という高度な認知タスクを題材として, 人間の脳神経システムと機械学習システムの内部処理メカニズムを比較分析する学際的研究である.
表面的な出力結果ではなく, 両者の根本的な処理方式の相違と共通点を明らかにし, 認知科学, 脳神経科学, 機械学習の各分野への相互的知見提供を目的とする.
認知科学的観点からはGileのエフォートモデルとSeeberの認知負荷モデルを中心として同時通訳の複雑性を検討し, 脳神経科学的観点からはfMRI, EEGなどの神経画像研究から基底核や前頭前野の活動パターンと神経可塑性を分析し, 情報工学的観点からはEnd-to-End同時音声翻訳システムとTransformerアーキテクチャの技術発展を詳述した.
比較分析の結果, 並列処理では人間が真の認知的並列処理を実現するのに対し機械は高速逐次処理に依存し, 戦略選択では人間が文脈依存的適応を行うのに対し機械は固定ポリシーを採用し, エネルギー効率では人間が20W程度で動作するのに対し機械は数百から数千Wを消費するという根本的相違が明らかになった.
一方で階層的処理, 注意機構, 予測機能において両者の収束的アプローチも確認された.
これらの知見から, 神経科学研究には機械学習の階層的処理を参考とした脳内メカニズムの詳細解明を, 機械学習システム開発には人間の並列処理と動的戦略選択を模倣した改善を提案した.
本研究は人間と機械の相互学習による新たな研究パラダイム確立への理論的基盤を提供する.