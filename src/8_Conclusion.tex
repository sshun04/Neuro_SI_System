\section{Conclusion}

\subsection{研究成果の総括}

本研究では, 同時通訳という高度な認知タスクを題材として, 人間の脳と機械学習システムの処理メカニズムを比較検討した.
認知科学的観点からは Seeber の認知負荷モデルを中心として, 脳神経科学的観点からは神経画像研究の知見を整理し, 情報工学的観点からは最新の End-to-End 同時通訳システムを分析した.

この比較から, 人間の真の並列処理と機械の高速逐次処理, 人間の文脈依存的戦略選択と機械の固定ポリシー, 人間の神経効率性と機械の高電力消費という具体的な相違点が明らかになった.
同時に, 階層的処理, 注意機構, 予測機能における類似性も確認された.
これらの知見により, 脳神経科学研究において機械学習システムの階層的アーキテクチャを参考にした詳細な処理段階のマッピングの重要性が示唆された.
また, 機械学習システム開発において人間の並列処理, 動的戦略選択, 神経効率性を参考にした改善の方向性が明らかになった.

\subsection{研究の限界と今後の課題}

本研究では, 人間の脳と機械学習システムの比較検討からインサイトを見出すところまでを行ったが, 各インサイトについての妥当性の検証は行えていない.
この比較研究により得られた知見は理論的な示唆にとどまり, 実証的な検証が必要である.

\subsection{今後の検証課題}

今後のステップとして, 提案された各インサイトが妥当なのかという点を神経科学側では脳計測, AI側ではモデルの実装や学習と評価などを通じて実際に検証する必要がある.

神経科学的検証においては, 本研究で提案された階層的処理段階の仮説を高時間分解能 EEG や MEG を用いて実証する必要がある.
特に, 機械学習システムで観察される注意重みの動的変化と, 同時通訳中の人間の注意配分パターンの対応関係を定量的に検証することが重要である.
また, 人間の並列処理と機械の逐次処理の違いについて, 同時性期間中の脳活動パターンの詳細な時間的解析により確認する必要がある.

機械学習システム側では, 人間の認知制約を模倣した容量制限付きメモリアーキテクチャの実装と評価が必要である.
また, 文脈依存的な動的戦略選択機構を実装し, 従来の固定ポリシーとの性能比較を行うことで, 人間の戦略選択の有効性を検証できる.
さらに, 神経効率性に着想を得た選択的活性化機構により, エネルギー効率の改善効果を定量的に測定する必要がある.

これらの検証研究により, 本研究で提案された理論的枠組みの妥当性が確認されれば, 人間と AI の相互学習による新たな研究パラダイムの確立につながると期待される.

