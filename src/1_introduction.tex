\section{Introduction}

本研究は, AI側で行われているアプローチの進化と, 人間の脳の解明を目指す認知科学的・脳神経科学的な研究の双方を比較・検討する立場を取る.
両分野の研究成果を横断的に分析することにより, AI と人間の能力解明の双方にとって有意義なインサイトを発見することを目的としている.

現代のAI技術の基盤を振り返ると, 興味深い歴史的経緯が浮かび上がってくる.
もともと現在注目されているAI技術の根幹であるニューラルネットワークの仕組みは, かつて人間の神経系をモデルとして再現されたものであった.
しかし, その後の数十年間で, ニューラルネットワークをはじめとするAI研究は生物学的な神経系とは大きく異なる独自の進化と発展を遂げ, 今日に至っている.

現代において, AIブームが人工知能への大きな期待とそれに対する社会的な関心を爆発的に高めていることは間違いない.
人工知能が人間を超えるのではないかという期待が社会全体で高まっているのが現状である.
人工知能研究の黎明期から一貫して存在する一つの根底的な考え方として, 脳はコンピューターの一種にすぎないという機械論的な世界観がある.
この観点では, 人工知能システムをどのように設計するかという内部的なプロセスは重要ではなく, 最終的に人間のような振る舞いができれば良いという結果主義的なアプローチが取られる.

昨今の様々な分野におけるAI技術の発展に伴い, その性能評価と人間との比較が盛んに行われている.
これらの評価において主流となっている手法と指標は, 入力に対してどのような出力を産出するかという表層的な入力--出力関係に焦点を当てたものである.
このような評価アプローチは, 内部的な処理メカニズムではなく, 最終的な結果の質のみを重視する特徴を持っている.

筆者は, このような結果重視のアプローチを全面的に否定するものではない.
特定のタスクを自動化し, 人間よりも効率的な形で機械的に処理することを目指すという点において, 極めて実用的で有益なアプローチであることは確かである.
実際に多くの産業分野でAI技術が導入され, 生産性の向上に寄与している事実は無視できない.

しかし, ここで重要な誤解を避けなければならない点がある.
現在のAI技術の高性能化は, 決してコンピューターが真の意味での知能を獲得したことを意味するものではないということである.
この点については, AI研究に携わる基本的な研究者であれば, 当然理解している前提事項である.
現在のAI システムは, あくまでも特定のタスクに特化した高度な統計的パターン認識システムであり, 人間のような汎用的な知能とは本質的に異なるものである.

一方で, 社会全般においては, AI技術に対して過度に楽観的な期待を抱く傾向が見られることも事実である.
人間の脳を超えるような万能な知能システムが近い将来実現されるという夢想的な期待を抱く人々が少なくない現状がある.
このような期待と現実のAI技術の間には, 依然として大きなギャップが存在している.

このような背景を踏まえ, 本研究では, 人間にとって非常に困難であり, 同時にAI技術にとってもまだ完全には解決されていない高難度タスクである同時通訳に注目する.
同時通訳は, 音声による同時通訳という複雑な認知的タスクであり, 人間とAI双方にとって極めて挑戦的な領域である.
本研究では, 人間の同時通訳タスクにおける認知的・神経科学的メカニズムと, AI システムにおける計算的アプローチを詳細に比較分析する.
人間が採用している戦略的アプローチとAI技術において実装されているアプローチの間にどの程度の差異が存在し, どの部分において相互に補完し合える要素があるのかを解明することを主題とする.

同時通訳(simultaneous interpreting; SI)は、原発話(Source Language; SL)の終了を待たずに、対訳(Target Language; TL)を同時に産出するタスクである.
SIは聞く,理解する, 翻訳する, 発話するといった複数のタスクを同時に行う必要があり, そのため通訳者には高い認知的負荷がかかる. 
同時通訳が要求される現場の多くは国際会議や企業の重役の会議など遅延や誤植が許容されない環境である.
プロの同時通釈者はこの厳しい時間的制約と精度の要求に対応するため,様々な戦略を取ることで認知的負荷を抑制しつつ, 通訳を行っている.[引用するよ~]

同時通訳タスクを同時通訳という高度なタスクを人がどのように解決し、高いパフォーマンスを発揮しているのか。この点を認知科学的観点からモデル化し、解明しようとする試みは古くから行われてきた。認知科学的観点からモデル化する試みはかねてから行われてきた.
Gllie[引用するよ〜]はカーネマン[引用するよ〜]の単一資源理論(single resource theory)に基づいて, SIを聞く・分析する,生産する,記憶する,調整するという4つの努力 (Effort)の合計として捉える努力モデル(Effort Model; EM)を提唱した.
Kilian Sebauer[引用するよ~]はウィッケンス[引用するよ〜]の多重資源モデル(Multiple Resource Model)に基づいて, SIを言語語理解タスクと言語生成タスクのリアルタイムな組み合わせとして捉え, 各タスク間の構造的な類似性によって生じる干渉と認知不可の変化を説明する認知負荷モデル(Cognitive Load Model; CLM)を提唱した.

同時通訳における課題の1つに言語の文法構造が異なる言語間で速度と精度を維持しながら通訳を行うことがある.
これは主に構文上の非対称性と,それに伴う認知負荷の増加に起因する.
英語のような主語-動詞-目的語 (SVO) 構造の言語と, ドイツ語や日本語のような主語-目的語-動詞 (SOV) 構造の言語間の通訳では目的語と動詞の位置の違いから,動詞または目的語の発話を待たざるを得ない状況が生じます.
Kilian Sebauer[引用するよ~]は認知負荷モデルにおいて, この構文的非対称性への対処として通訳者は待機 (waiting),時間稼ぎ (stalling),チャンキング (chunking),予測 (anticipating)の4つの戦略を用いると提唱した.
認知不可モデルではこれら4つの戦略がそれぞれ通訳者の認知的負荷にどのような影響を与えるかを具体的に示し定量化を試みている.

脳神経科学的な研究では, SIタスク中における脳の活動を観察することで, 通訳を行う神経基盤的なメカニズムを明らかにしている.
同時通訳は前頭前野、基底核、側頭葉、頭頂葉を中心とする広域神経ネットワークを動員し、熟練通訳者ではより効率的で集約的な神経活動パターンを示すことが2020年以降の神経画像研究により明らかになった. [引用するよ~]
別の研究ではSeeberの認知負荷モデルの各構成要素(P-C-R-S)は特定の脳領域と対応し、SOV-SVO語順変換では特に前頭前野と基底核の活動増加が認められる.[引用するよ~]
言語ペアの類型的距離が神経適応メカニズムに影響を与えることが判明しており、言語間の構造的非対称性を処理する際の神経基盤的な知見が蓄積されている.[引用するよ~]

計算機による同時通訳システムは人に変わる手法として同時通訳を導入できる状況を増やすためにニーズがある.
情報工学の分野において提唱されてきた同時通訳システムは一般に同時通訳タスクを音声認識, 機械翻訳, 音声合成の3つ要素技術組み合わせによって実現するものである [引用するよ~]
また最近では上記3つの要素をTranformer[引用するよ~]を用いた機械学習モデルにより1つのストリーム化された処理として行うEnd to Endの同時通訳システムが注目されている [引用するよ~]

脳神経科学的な観点と情報工学的な観点を横断的に検討することが必要である.
今後のトレンドとして機械学習モデルを用いた同時通訳システムが色々と出てくるだろう.
しかしこれらのシステムは高性能な機械学習モデルを用いることを前提としており, 膨大なパラメーター数を持ち計算コストが高い.
人の脳内の神経細胞の数やその結合とAIのパラメーターを単純比較することは難しいが, 非常に短絡的な比較においてはAIの方が数十倍の電力エネルギーを必要としていることは確かである.
その点において, 人の脳内の神経細胞の数やその結合を模倣することで, 同時通訳システムの計算コストを削減することができると考えられる.
これは同時通訳タスクにとどまらず, 様々なタスクにおいて同様の効果が期待できると考えられる.