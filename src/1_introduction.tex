\section{Introduction}

同時通訳(simultaneous interpreting; SI)は、原発話(Source Language; SL)の終了を待たずに、対訳(Target Language; TL)を同時に産出するタスクである.
SIは聞く,理解する, 翻訳する, 発話するといった複数のタスクを同時に行う必要があり, そのため通訳者には高い認知的負荷がかかる. 
同時通訳が要求される現場の多くは国際会議や企業の重役の会議など遅延や誤植が許容されない環境である.
プロの同時通釈者はこの厳しい時間的制約と精度の要求に対応するため,様々な戦略を取ることで認知的負荷を抑制しつつ, 通訳を行っている.[引用するよ~]

同時通訳タスクを認知科学的観点からモデル化する試みはかねてから行われてきた.
Gllie[引用するよ〜]はカーネマン[引用するよ〜]の単一資源理論(single resource theory)に基づいて, SIを聞く・分析する,生産する,記憶する,調整するという4つの努力 (Effort)の合計として捉える努力モデル(Effort Model; EM)を提唱した.
Kilian Sebauer[引用するよ~]はウィッケンス[引用するよ〜]の多重資源モデル(Multiple Resource Model)に基づいて, SIを言語語理解タスクと言語生成タスクのリアルタイムな組み合わせとして捉え, 各タスク間の構造的な類似性によって生じる干渉と認知不可の変化を説明する認知負荷モデル(Cognitive Load Model; CLM)を提唱した.

同時通訳における課題の1つに言語の文法構造が異なる言語間で速度と精度を維持しながら通訳を行うことがある.
これは主に構文上の非対称性と,それに伴う認知負荷の増加に起因する.
英語のような主語-動詞-目的語 (SVO) 構造の言語と, ドイツ語や日本語のような主語-目的語-動詞 (SOV) 構造の言語間の通訳では目的語と動詞の位置の違いから,動詞または目的語の発話を待たざるを得ない状況が生じます.
Kilian Sebauer[引用するよ~]は認知負荷モデルにおいて, この構文的非対称性への対処として通訳者は待機 (waiting),時間稼ぎ (stalling),チャンキング (chunking),予測 (anticipating)の4つの戦略を用いると提唱した.
認知不可モデルではこれら4つの戦略がそれぞれ通訳者の認知的負荷にどのような影響を与えるかを具体的に示し定量化を試みている.

脳神経科学的な研究では, SIタスク中における脳の活動を観察することで, 通訳を行う神経基盤的なメカニズムを明らかにしている.
同時通訳は前頭前野、基底核、側頭葉、頭頂葉を中心とする広域神経ネットワークを動員し、熟練通訳者ではより効率的で集約的な神経活動パターンを示すことが2020年以降の神経画像研究により明らかになった. [引用するよ~]
別の研究ではSeeberの認知負荷モデルの各構成要素(P-C-R-S)は特定の脳領域と対応し、SOV-SVO語順変換では特に前頭前野と基底核の活動増加が認められる.[引用するよ~]
言語ペアの類型的距離が神経適応メカニズムに影響を与えることが判明しており、言語間の構造的非対称性を処理する際の神経基盤的な知見が蓄積されている.[引用するよ~]

計算機による同時通訳システムは人に変わる手法として同時通訳を導入できる状況を増やすためにニーズがある.
情報工学の分野において提唱されてきた同時通訳システムは一般に同時通訳タスクを音声認識, 機械翻訳, 音声合成の3つ要素技術組み合わせによって実現するものである [引用するよ~]
また最近では上記3つの要素をTranformer[引用するよ~]を用いた機械学習モデルにより1つのストリーム化された処理として行うEnd to Endの同時通訳システムが注目されている [引用するよ~]


脳神経科学的な観点と情報工学的な観点を横断的に検討することが必要である.
今後のトレンドとして機械学習モデルを用いた同時通訳システムが色々と出てくるだろう.
しかしこれらのシステムは高性能な機械学習モデルを用いることを前提としており, 膨大なパラメーター数を持ち計算コストが高い.
人の脳内の神経細胞の数やその結合とAIのパラメーターを単純比較することは難しいが, 非常に短絡的な比較においてはAIの方が数十倍の電力エネルギーを必要としていることは確かである.
その点において, 人の脳内の神経細胞の数やその結合を模倣することで, 同時通訳システムの計算コストを削減することができると考えられる.
これは同時通訳タスクにとどまらず, 様々なタスクにおいて同様の効果が期待できると考えられる.

本研究では, 構造的非対称性を持つ言語間のシステム同時通訳に着目し, 
1. 認知科学的観点としてKilian Sebauer[引用するよ~]の認知負荷モデルとそれに対応する脳神経基盤的な情報処理の仕組みをマッピングする
2. 1に基づき精度と速度を維持しながら通訳を生成するための情報工学的なモデルを提案し, その有効性を検証する.
3. 上記の一連を通じて,認知科学的観点脳神経科学的な観点が情報工学的な計算機の改善において効果的であることを主張する.