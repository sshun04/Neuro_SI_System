\section{Introduction}

本研究は, 同時通訳という高度な認知タスクを題材として, 人間の脳神経システムと機械学習システムの内部処理メカニズムを詳細に比較分析する学際的研究である.
表面的な出力結果の類似性に惑わされることなく, 両者の根本的な処理方式の相違と共通点を明らかにし, 人間の脳神経基盤と機械学習システムの異なる2つの学問分野間へ相互に知見を提供することを目的とする.

現代のAI技術の基盤を振り返ると, 興味深い歴史的経緯が浮かび上がってくる.
もともと現在注目されているAI技術の根幹であるニューラルネットワークの仕組みは, かつて人間の神経系をモデルとして再現されたものであった.
しかし, その後の数十年間で, ニューラルネットワークをはじめとするAI研究は生物学的な神経系とは大きく異なる独自の進化と発展を遂げ, 今日に至っている.

現代において, AIブームが人工知能への大きな期待とそれに対する社会的な関心を爆発的に高めていることは間違いない.
人工知能が人間を超えるのではないかという期待が社会全体で高まっているのが現状である.
人工知能研究の黎明期から一貫して存在する一つの根底的な考え方として, 脳はコンピューターの一種にすぎないという機械論的な世界観がある.
この観点では, 人工知能システムをどのように設計するかという内部的なプロセスは重要ではなく, 最終的に人間のような振る舞いができれば良いという結果主義的なアプローチが取られる.

昨今の様々な分野におけるAI技術の発展に伴い, その性能評価と人間との比較が盛んに行われている.
これらの評価において主流となっている手法と指標は, 入力に対してどのような出力を産出するかという表層的な入力--出力関係に焦点を当てたものである.
このような評価アプローチは, 内部的な処理メカニズムではなく, 最終的な結果の質のみを重視する特徴を持っている.

筆者は, このような結果重視のアプローチを全面的に否定するものではない.
特定のタスクを自動化し, 人間よりも効率的な形で機械的に処理することを目指すという点において, 極めて実用的で有益なアプローチであることは確かである.
実際に多くの産業分野でAI技術が導入され, 生産性の向上に寄与している事実は無視できない.

しかし, ここで重要な本質的理解が必要な点がある.
現在のAI技術の高性能化は, 統計的パターンマッチングの精緻化であり, 人間の脳が持つ概念的理解や文脈依存的推論とは根本的に異なる処理メカニズムに基づいている.
人間の脳は約20Wという低電力で認知的並列処理を実現し文脈依存的な戦略選択を行うのに対し, 現在のAIシステムは数百から数千Wの電力を消費し高速逐次処理と固定的ポリシーに依存している.
この処理方式の根本的相違こそが, 真の知能と高度な統計処理システムを区別する鍵となる.

一方で, 社会全般においては, AI技術に対して過度に楽観的な期待を抱く傾向が見られることも事実である.
人間の脳を超えるような万能な知能システムが近い将来実現されるという夢想的な期待を抱く人々が少なくない現状がある.
このような期待と現実のAI技術の間には, 依然として大きなギャップが存在している.

このような背景を踏まえ, 本研究では同時通訳という人間にとってもAI技術にとっても極めて挑戦的なタスクを研究対象として選択した.
同時通訳は真の並列処理能力, 高度な文脈理解, 動的な戦略選択, エネルギー効率性という4つの側面において, 人間と機械の能力差が顕著に現れる理想的な比較対象である.
本研究では, 人間の脳神経基盤における認知的並列処理メカニズムと機械学習システムの高速逐次処理アーキテクチャを体系的に比較し, 両者の根本的相違と収束的側面を明らかにする.
さらに, この比較分析から得られる知見を神経科学研究の新たな方法論と機械学習システムの改善提案という双方向の発展に活用する可能性を探求する.

同時通訳(simultaneous interpreting; SI)は、原発話(Source Language; SL)の終了を待たずに、対訳(Target Language; TL)を同時に産出するタスクである.
SIは聞く,理解する, 翻訳する, 発話するといった複数のタスクを同時に行う必要があり, そのため通訳者には高い認知的負荷がかかる. 
同時通訳が要求される現場の多くは国際会議や企業の重役の会議など遅延や誤植が許容されない環境である.
プロの同時通釈者はこの厳しい時間的制約と精度の要求に対応するため,様々な戦略を取ることで認知的負荷を抑制しつつ, 通訳を行っている \cite{seeber2011cognitive} .

同時通訳という高度なタスクを人がどのように解決し, 高いパフォーマンスを発揮しているのかという点を認知科学的観点からモデル化し, 解明しようとする試みは古くから行われてきた.
Gile \cite{gile1995regards, gile1997conference} は Kahneman \cite{kahneman1973attention} の単一資源理論 (single resource theory) に基づいて, SIを聞く・分析する, 生産する, 記憶する, 調整するという4つの努力 (Effort) の合計として捉える努力モデル (Effort Model; EM) を提唱した.
Seeber \cite{seeber2011cognitive} は Wickens \cite{wickens1984processing, wickens2002multiple} の多重資源モデル (Multiple Resource Model) に基づいて, SIを言語理解タスクと言語生成タスクのリアルタイムな組み合わせとして捉え, 各タスク間の構造的な類似性によって生じる干渉と認知負荷の変化を説明する認知負荷モデル (Cognitive Load Model; CLM) を提唱した.

同時通訳における課題の1つに言語の文法構造が異なる言語間で速度と精度を維持しながら通訳を行うことがある.
これは主に構文上の非対称性と,それに伴う認知負荷の増加に起因する.
英語のような主語-動詞-目的語 (SVO) 構造の言語と, ドイツ語や日本語のような主語-目的語-動詞 (SOV) 構造の言語間の通訳では目的語と動詞の位置の違いから, 動詞または目的語の発話を待たざるを得ない状況が生じる.
Seeber \cite{seeber2011cognitive} は認知負荷モデルにおいて, この構文的非対称性への対処として通訳者は待機 (waiting), 時間稼ぎ (stalling), チャンキング (chunking), 予測 (anticipating) の4つの戦略を用いると提唱した.

脳神経科学的な研究では, SIタスク中における脳の活動を観察することで, 通訳を行う神経基盤的なメカニズムを明らかにしている.
同時通訳は前頭前野, 基底核, 側頭葉, 頭頂葉を中心とする広域神経ネットワークを動員し, 熟練通訳者ではより効率的で集約的な神経活動パターンを示すことが神経画像研究により明らかになった \cite{hervais2015fmri, vandeputte2018anatomical} .
別の研究では Seeber の認知負荷モデルの各構成要素は特定の脳領域と対応し, SOV-SVO 語順変換では特に前頭前野と基底核の活動増加が認められる \cite{yagura2021selective} .
言語ペアの類型的距離が神経適応メカニズムに影響を与えることが判明しており, 言語間の構造的非対称性を処理する際の神経基盤的な知見が蓄積されている \cite{lin2018costly, ishizuka2024two} .

計算機による同時通訳システムは人に代わる手法として同時通訳を導入できる状況を増やすためにニーズがある.
情報工学の分野において提唱されてきた同時通訳システムは一般に同時通訳タスクを音声認識, 機械翻訳, 音声合成の3つの要素技術の組み合わせによって実現するものである \cite{doi2024evaluation} .
また最近では上記3つの要素を Transformer \cite{vaswani2017attention} を用いた機械学習モデルにより1つのストリーム化された処理として行う End-to-End の同時通訳システムが注目されている \cite{liu2024recent, ma2024nast, zhang2024streamspeech} .

本研究の学際的アプローチは, 従来の分野別研究では見過ごされてきた重要な知見をもたらす可能性を秘めている.
人間の脳が実現する認知的並列処理, 20W程度の超低電力動作, 文脈依存的な戦略選択といった優位性を機械学習システムに統合することで, より効率的で適応的な同時通訳システムの実現が期待される.
同時に, 機械学習システムで実証された階層的処理の有効性, 注意機構の重要性, 分散表現の利点から逆算的に人間の脳神経基盤を理解する新たな研究パラダイムの構築が可能となる.
この双方向の知見交換により, 同時通訳研究のみならず認知科学, 脳神経科学, 機械学習の各分野に対して革新的な示唆を提供することを目指す.