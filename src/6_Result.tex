\section{Results}

\subsection{Neural-Inspired Cognitive Load Analysis}

神経基盤に基づく実装により、Anticipation戦略が最も低い平均負荷(7.1/10)を示し、これは内側前頭前野による効率的な予測処理の実装効果と考えられる。Waiting戦略(11.2/10)は右下前頭回の抑制制御コストが高く、Stalling戦略(12.1/10)では基底核の時間制御における計算オーバーヘッドが確認された。

EEG研究で示された前頭シータパワー(4-8Hz)の変調パターンを模倣した負荷測定では、熟練通訳者の神経効率性パターンとの相関係数0.78を達成した。特に、Elmer \& Kühnis (2016)が報告した左背側経路の強化された接続性を模倣したAnticipation戦略では、レイテンシが0.9秒と大幅に改善された。 

\subsection{Neural Efficiency vs. AI Parameter Efficiency}

\subsection{Strategy-Specific Neural Correlations}
