\section{Theoretical Framework}

Neuroscientific research (using fMRI, EEG, etc.) has begun to reveal how the brain copes with the extreme demands of SI – in particular, the cognitive control networks that enable interpreters to listen and speak simultaneously while managing high load. Modern brain imaging confirms that SI engages not only classical language areas but also executive-control regions, especially under challenging conditions like word-order asymmetry.

人間の脳内では同時通訳を行う際に
高度かつ複合的な認知制御を行う

## 現在の機械同時通訳システムのアプローチと人間の同時通訳のアプローチのギャップ
過去の先行研究から、機械による同時通訳システムと人間の同時通訳のアプローチには大きなギャップがあることが明確です。まず一つ目として、自明で明確なことは、機械同時通訳システムが同時通訳を前から進むタスク、つまり逐次的にインプットを処理することを目的としている点です。

特にストリーミング処理によって同時通訳のリアルタイム性を担保しようとするタスクにおいては、
音声が一方向的に入ってくることを想定しており、それらを徐々にアウトプットしていくことを考えています。
アテンション機構を用いたものに関しても、ユーザーの入力はアテンション機構の入力に基づき、
通訳を産出するかどうかという点において動的にタイミングを決定する機構であるものの、
これはあくまで同時通訳に関して前からタスクを順番に処理する、つまり時間的には重複のない順次時間の流れに沿って行うタスクであるということを意味しています。

一方で、実際の過去の認知科学的観点からの研究、あるいは脳基盤を見た研究では、
同時通訳は人間において耳で言語を聞きながら言語制御を行い、ソース言語の意味を理解し、
それを対象言語へと翻訳し、発話の制御を行い発話を行うというプロセスを含みます。そしてまた、
そこに合わせてエラーの監視や通訳を自然にするための様々な認知科学的戦略を用い、時間に重複がある形で情報を保持しながら、
脳内で非常に高度な認知処理を行い、ストリーム情報の保持など複合的なタスクを同時並行で行うタスクであるという点が明らかになっています。

この大きな差異は、この同時通訳というタスクを複数のタスクが同時並列で行われる複合的なタスクと捉えるか、
それとも発話された処理を前から順番に処理していく、その精度を上げるアプローチであるというふうに解釈するシステム通訳の現在のアプローチには大きな違いがあり、
現在プロフェッショナルの最も精度の高い人間が行っているような同時通訳を超える、あるいは再現する、
あるいはさらなる人の同時通訳の自然さ、柔軟さを受け継ぎながら、よりシステムの持てるメモリーや計算能力といったそういった限界がない点を加味したより高度なモデルを作る際には、
現在の同時通訳システムにおけるアプローチでは必ず限界が来ると考えている。

機械システムにおけるエンドトゥエンドの同時通訳システムは、処理を音声、言語理解、音声産出、そして認知制御といった高度なタスクからなる実際の人間の同時通訳の処理に、
AI側の処理も近づけるアプローチと捉えることができる。しかし、この研究はまだ初期段階であり、実装に時間がかかると思われる。また、昨今の大規模言語モデルの向かう先が、処理を出すためには大型化、つまりパラメータ数を増やし計算速度を増やすという方向に向いているという観点においては、
実際には同時通訳というリアルタイム性が高く求められるような環境において、その早い実行精度を保つために必要となる計算リソースが膨大となることは容易に想像がつくものである

ここでは将来のそういった課題を踏まえ、より効率が人間の脳と同時通訳を参照としながら、
同時通訳に関わる脳の処理基盤、そこにあるネットワーク、そして認知科学的な解明といった点を、
実際の人に根ざした観点からより効果的なシステム、それをシステム通訳に落とし込むためのアプローチ、
そのアーキテクチャを提唱していく。


## 実装に向けた理論構築
まず本実装における重要な点として、同時通訳において脳内で発生する処理タスクがどのようなものかという点である。ここは先行研究をもとに分解していってみる。

これは聴覚での音声理解から始まり、その意味の理解、そしてターゲット言語への翻訳、その際の言語の制御、そしてエラーの監視、そして発話の制御、発話後のエラーの監視といった具合に分けられると想像される。

実際の同時通訳における各タスクの課題を分類してみる。
言語の判定について、これは実際の同時通訳の現場、人による同時通訳でも、ソースランゲージが何であり、ターゲットランゲージが何かということは事前情報として定められていることが多い。
そのため、この点は固定パラメータとして、あらかじめ言語の方向、その会議内で扱われる言語の種別という点は往々にして明らかなケースがほとんどであるため、この点は問題ないと固定値として問題ないと考える。