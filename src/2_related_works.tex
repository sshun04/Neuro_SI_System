\section{Background and Related Work}

\subsection{Cognitive Models of Simultaneous Interpretation}

Gileの努力モデル(Effort Model; EM)は同時通訳における認知的プロセスを理解する上で基礎的な枠組みを提供したが, いくつかの限界があった.
Kahneman[引用するよ~]の単一資源理論に基づくこのモデルは, 全ての認知タスクが1つの未分化な資源プールを競合すると仮定している.
しかし, この理論では完璧な時分割(perfect time-sharing)現象を説明できず, タスク構造の変化が異なる干渉度を生み出すことも説明できない.
[引用するよ~]

Seeberの認知負荷モデル(Cognitive Load Model; CLM)はGileの努力モデルの限界を克服するため, Wickens[引用するよ~]の多重資源理論に基づいて開発された革新的なアプローチである.
従来のGileモデルがKahnemanの単一資源理論に基づき「認知システム全体で1つの処理容量プールを共有する」と仮定していたのに対し, Seeberのモデルは「認知システム内に複数の専門化された処理資源が存在する」という多重資源の概念を採用している.

このモデルの核心は, 同時通訳を言語理解タスク(listening and comprehension)と言語生成タスク(production)のリアルタイムな組み合わせとして捉え, 構造的に類似したタスク間でより強い干渉が生じるという予測にある.
例えば, ドイツ語の聴解と英語の発話は両方とも「聴覚-言語的処理」に分類されるため, 視覚的情報処理と聴覚的情報処理の組み合わせよりも高い干渉度を示す.
具体的には, 通訳者がドイツ語の複雑な従属節を聞きながら英語で発話する場合, 両タスクが同じ認知資源(聴覚-言語チャンネル)を競合するため, 認知負荷が大幅に増加する.

このモデルの最大の革新性は, 原言語の入力特性と目標言語の出力特性の両方を統合的に考慮できる点にある.
需要ベクトル(demand vectors)と呼ばれる多次元的指標を用いて, 知覚処理(perceptual processing), 認知処理(cognitive processing), 応答処理(response processing), さらに情報保持(storage)という4つの処理段階における局所的認知負荷を定量的に分析する.
これにより, 文の特定の時点(例:動詞末構造における動詞出現前後)で認知負荷がどの程度変化するかを時間軸に沿って詳細に予測し, 通訳戦略の選択が認知負荷に与える影響を数値的に評価することが可能となった.
[引用するよ~]

このモデルにおける4つの処理段階は, それぞれ同時通訳の異なる認知作業に対応している.
知覚処理(P: Perceptual Processing)は音響信号の受信から語彙認識までの初期段階を担う.
具体的には, 原言語(例:ドイツ語)の発話を聞く, 音韻パターンを識別する, 個々の単語を認識するといったタスクが含まれる.
例えば, 「die Delegierten」という音響入力を聞いて「代表団」という語彙項目として認識する過程がこれに該当する.

認知処理(C: Cognitive Processing)は言語理解の中核となる統語・意味分析を行う.
原言語の統語構造を解析し(例:従属節の識別), 意味を解釈し(例:行為主と行為の関係把握), 文脈情報を統合する作業が含まれる.
「dass die Delegierten ihre Entscheidung treffen」において, 「dass」節が従属節であること, 「Delegierten」が主語で「treffen」が述語であることを理解し, さらに「決定を下す」という概念的意味を構築する処理がこれに相当する.

応答処理(R: Response Processing)は目標言語での言語産出を制御する.
目標言語(例:英語)での語彙選択, 統語構造の組み立て, 音韻符号化, 実際の発話運動制御が含まれる.
例えば, 「代表団が決定を下す」という概念を「the delegates make a decision」として英語で表現し, 適切な語順で発話する過程がこれに該当する.

保存処理(S: Storage)はワーキングメモリでの情報保持を管理する.
特に動詞末構造の処理において, 主語や目的語などの文構成要素を一時的に保持し, 動詞が出現するまで維持する作業が中心となる.
「dass die Delegierten ihre Entscheidung nach einer langen Debatte treffen」において, 「die Delegierten」と「ihre Entscheidung」を動詞「treffen」が出現するまでワーキングメモリに保持し続ける処理がこれに相当する.
[引用するよ~]

言語間の構造的非対称性に対処するため, Seeberは通訳者が用いる4つの認知戦略を特定した.
第一の戦略である待機(waiting)は, より多くの原言語情報を得るために目標言語の産出を一時停止する戦略である.
この戦略により通訳者は認知負荷を一時的に軽減できるが, 情報をワーキングメモリに保持する必要があり, 下流での認知負荷の大幅な増加を招く可能性がある.
[引用するよ~]

第二の戦略である時間稼ぎ(stalling)は待機と同様に時間を稼ぐことを目的とするが, 沈黙の代わりに"中性的な埋め草"を産出する.
この戦略は聞き手や通訳者自身の不快感を軽減するが, 埋め草の符号化と産出が理解プロセスと重複するため処理の複雑さを増す.
また, 待機戦略と同様に通訳者の遅延時間(lag)を蓄積し, 全体的な認知負荷の増加をもたらす.
[引用するよ~]

第三の戦略であるチャンキング(chunking)は, 文を完全に展開されるのを待たずに符号化できる小さな断片に入力を分割する戦略である.
この戦略では原言語入力を即座に統合・符号化できるが, 引数間を関連付ける主動詞の不在により断片を下流で繋ぎ合わせる必要が生じる.
結果として時間的に遅延した認知負荷の増加と, しばしば不自然で"言語に暴力を加える"ような構文を生成する可能性がある.
[引用するよ~]

第四の戦略である予測(anticipation)は, 話者による発話に先立って原談話の一部を予測する能力である.
この戦略は推論処理("推測")に伴う認知資源を除いて, 認知負荷をベースライン値に近く維持することができる.
また, ベースライン値に近い遅延時間で通訳を完了でき, 他の戦略で見られるスピルオーバー効果を回避できる理想的な解決策である.
しかし, 予期しない動詞による"驚き"のリスクが伴い, 文の重要な意味的・文体的要素を損なう危険性がある.
[引用するよ~]

Seeberの研究では認知負荷の定量化を実現するため, 瞳孔反応測定法(pupillometry)という心理生理学的手法を採用した.
この手法は認知活動に伴う瞳孔径の変化を測定することで, 意識的制御が困難な客観的な認知負荷指標を提供する.
Wickensの多重資源理論に基づいた干渉係数(conflict coefficients)と組み合わせることで, 各通訳戦略における局所的認知負荷の変化を数値的に表現することを試みている.
[引用するよ~]

\subsection{Affect of Grammatical Structure on Cognitive Load during SI by professional interpreters}

Seeber \& Kerzel (2012) \cite{seeber2012cognitive}は瞳孔測定法(pupillometry)を用いてCLMの予測を実証的に検証した.
ドイツ語の動詞末構造から英語への同時通訳実験において, 10名のプロ通訳者を対象とした心理生理学的測定を実施した.
実験では動詞初期構造(verb-initial)をベースラインとし, 動詞末構造(verb-final)との認知負荷を比較した.
瞳孔径の変化を250Hzで連続測定することで, 同時通訳中の局所的認知負荷をリアルタイムで定量化することに成功した.

実験結果では, 動詞末構造の通訳時に文末付近(Period of Interest 4)で瞳孔径が有意に拡大し, 認知負荷の顕著な増加が確認された.
この負荷増加は予測された時点で現れ, CLMが示す「下流への負荷輸出(exported load)」現象を実証した.
また文脈あり条件では文脈なし条件と比較して認知負荷が軽減される傾向が観察され, 推論処理による負荷軽減効果が示唆された.
重要な点として, 実験中に認知的過負荷(cognitive overload)を示す急激な瞳孔収縮は観察されず, 通訳者が能力限界内で作業していることが確認された.
これらの結果は構造的非対称性が同時通訳に実質的な認知的コストを課すことを客観的に立証した.

Seeber (2013) \cite{seeber2013cognitive}は認知負荷測定手法の包括的な分析を行い, 瞳孔測定法の同時通訳研究における有効性を論じた.
心理生理学的手法としての瞳孔測定法は, 主観的方法や分析的方法と比較して客観性と時間分解能の点で優位性を持つ.
認知活動に伴う瞳孔拡大は刺激提示後300-500msで開始し, 交感神経系の自動的反応として意識的制御が困難である.
この手法により句レベルや文レベルでの局所的認知負荷変化を捉えることができ, 同時通訳の動的な認知プロセス解明に貢献している.

\subsection{Neural Correlates of Cognitive Load during Structural Asymmetry Processing}
神経科学的研究により, 同時通訳の高度な認知負荷に対して脳がどのように対処するかが明らかになりつつある. 
特に, 機能的MRI, EEG, 拡散テンソル画像などの脳画像技術を用いた研究は, 同時通訳が言語野のみならず一般的な実行制御に関わる脳領域も活性化させることを確認している.
これらの神経科学的知見は, SeeberのCLMが予測する構造的非対称性処理時の認知負荷増加に対する脳の対処メカニズムを理解する上で重要な示唆を提供している.

Hervais-Adelman et al. (2015) \cite{hervais2015fmri}の研究では, 機能的MRIを用いて50名の多言語話者における脳活動を測定した. 
実験協力者は同時通訳とより単純な復唱(シャドーイング)タスクを実行し, ジュネーブで研究された実験協力者には訓練された会議通訳者が含まれていた. 
通訳タスク(シャドーイングとの比較において)では, 音声理解と産出ネットワーク全体に加えて, 領域汎用的認知制御に関連する追加の脳領域が強く活性化した.

特に注目すべきは, 皮質下基底核の両側尾状核が同時通訳中に強い活性化を示したことである. 
尾状核は課題切り替え, 抑制制御, 複数課題の協調を担う領域として知られており, 通訳者が同時の聞き取りと発話を処理するために脳の汎用実行回路を動員していることを示唆している.
実際に Hervais-Adelman et al. は, 二言語言語制御のための脳ネットワークと非言語的実行制御のためのネットワークの間に顕著な重複を観察した. 
この発見は, 実行タスクにおいて見られる二言語話者の利点が通訳の集中的な言語制御練習に由来するという考えを支持するものである.

重要なことに, この研究では脳が同時入力と出力(同時通訳の「二重課題」側面)をどのように管理するかについても検討された. 
被殻(別の基底核構造)のfMRI信号は聞き取りと発話の重複持続時間を追跡し, 通訳者が聞きながら発話する時間が長いほど被殻の活性化が大きくなることが示された.
研究者らはこの発見を線条体制御システム内の機能分離として解釈した. 尾状核は言語処理の高次選択と協調を処理し, 被殻は二重課題条件下での発話のオンライン運動制御を処理する. 

\subsubsection{同時通訳経験による脳可塑性の変化}

同時通訳が認知制御の限界に挑戦するため, 研究者らは長期的な通訳経験が神経適応をもたらすかどうか, すなわち持続的な高認知負荷を処理するために脳がどのように「再配線」されるかを調査している.
近年の複数の研究では, 訓練された通訳者と経験の少ない二言語話者を比較し, 実行制御に関連する機能的・構造的脳差異を明らかにしている.

Van de Putte et al. (2018) \cite{vandeputte2018anatomical}の縦断的fMRI研究では, 9ヶ月間の集中訓練にわたって通訳学生を追跡した.
研究者らは通訳初心者の訓練生グループと, 対照群である二言語翻訳学生を訓練期間前後で比較した.
一般的課題での行動パフォーマンスに差がなかったにもかかわらず, 神経画像結果では明確な訓練誘導性脳変化が示された.
訓練後, 通訳群では対照群と比較して非言語実行制御課題(サイモン課題や課題切り替え)中に右角回と左上側頭回での活性化増加が見られた.
これらの領域は注意と言語処理に関連しており, 通訳訓練が一般的な認知制御課題でさえ脳反応を促進することを示唆している.

さらに拡散MRIを用いて, Van de Putte et al. は通訳群でのみ2つの重要な脳ネットワークでの構造的結合性強化を発見した:
(1) 前頭-基底核回路(前頭実行領域と尾状核を結ぶ)で, 領域汎用および言語特異的制御の両方に関与する
(2) 小脳と補足運動野(SMA)を結ぶネットワークで, 高負荷言語制御と運動協調に関与する
これらの神経可塑性変化は, 通訳訓練が文字通り脳の制御ネットワークを再形成し, 同時通訳が要求する「極限言語制御」により良く対処できるようにすることを示唆している.

\subsubsection{安静時脳結合性と前頭葉の超結合性}

通訳者の適応した脳の更なる証拠として, 安静時脳結合性の研究が挙げられる.
Klein et al. (2018) \cite{klein2018interpreter}は, 訓練された同時通訳者の安静時(課題を実行していない状態)でのEEGを記録し, 多言語対照群と比較した.
安静状態においてさえ, 通訳者は対照群よりも前頭脳領域間でより大きな機能的結合性を示した.
具体的には, EEG信号のグラフ解析により, 通訳者ではアルファ周波数帯域で左下前頭回(ブローカ野, 弁蓋部/三角部)と背外側前頭前野(DLPFC)の間に超結合性が発見された.

これらの前頭領域は言語産出と実行機能(ワーキングメモリ, 注意制御)に重要である.
安静時においてさえより強く相互接続されているという事実は, 持続的な神経適応を示唆している:
同時通訳の慢性的高認知要求が実行前頭領域間のコミュニケーション経路強化をもたらし, 困難な翻訳中に必要な迅速な切り替えと抑制を支援する.
実質的に, 通訳者の脳はマルチタスキングと制御のために「調律」されており, 同時通訳経験による前頭統合促進の神経的指紋を示している.

\subsubsection{聴覚-運動統合経路の機能的強化}

EEG研究では, 通訳者の脳が負荷下でどのように言語を処理するかについて機能的違いも実証されている.
Elmer \& Kühnis (2016) \cite{elmer2016functional}は, EEGを用いて同時通訳者と二言語対照群の脳の背側ストリーム(聴覚-運動統合経路)の活用を検討した.
実験協力者は翻訳の要求を近似する2言語混合聴覚意味決定課題を実行した.
Elmer and Kühnis は, 訓練により通訳者が翻訳の迅速な定式化を促進するために「音響-調音」経路(聴覚皮質と前頭発話領域を結ぶ)により強く依存すると仮説を立てた.

実際に, EEG結合解析では, 通訳者は対照群と比較して左聴覚皮質とブローカ野(背側ストリームの2つの主要ハブ)間でシータ帯域位相同期が有意に高いことが示された.
この機能的結合増加は, 通訳者の脳がより密接に聴取と発話領域を結合し, 聞いた単語からその翻訳準備への迅速な転換を可能にしていることを示唆している.
さらに通訳群内では, これらの神経結合測定値は経験と相関していた: より多くの訓練時間がより強い結合性をもたらし, より早い通訳訓練開始年齢もより大きな結合と相関していた.
この用量効果関係は, 脳が同時通訳要求に対して構造的・機能的に適応することを強化し, 本質的に知覚と産出システム間の「スループット」を改善している.

\subsubsection{経験レベル別の認知制御EEGマーカー}

最後に, 英語と日本語を扱う熟練通訳者と初心者の間で認知制御EEGマーカーの違いが観察されている.
Yagura et al. (2021) \cite{yagura2021selective}は, 2つのグループ(エキスパート通訳者(≥15年経験)対初心者(<1年))のEEG信号(40Hz聴覚定常状態応答)を測定した.
実験協力者は日本語から英語への同時通訳を, シャドーイング課題と比較して実行した.
聴取と発話の両立に重要な選択的注意能力に焦点が当てられた.

Yagura and colleagues は経験レベルと課題の間に有意な交互作用を発見した:
エキスパートは初心者と比較して, シャドーイングよりも通訳中により高い位相固定神経応答を示した.
実質的に, 注意処理のEEG測定(40Hz応答の試行間位相一貫性)は長年の同時通訳経験により促進された.
これは同時通訳の広範囲な練習が課題間での注意協調能力を向上させることを示唆しており, 通訳者が優れた実行制御を獲得するという考えと一致している.

このYagura et al. の研究では日本語→英語方向(SOVからSVO)を使用したため, 全実験協力者が構造的遅延の挑戦に直面した.
より経験豊富な通訳者は, 集中的注意と効率的な課題切り替えの神経的特徴に反映されるように, これらの挑戦をより良く処理することができた.
この発見は, 同時通訳経験が脳活動パターンを調節し, 通訳中の選択的注意を改善するという結論を支持している.

\subsubsection{初期神経画像研究からの知見}

同時通訳の初期神経画像研究も, 統語変換と二重課題要求の神経基盤に関する重要な洞察を提供している.
Rinne et al. (2000) による研究では, 通訳者の脳が単純な復唱よりも通訳中に左下前頭皮質(ブローカ野)と補足運動野(SMA)をより強く動員することが発見された.
これらの領域は統語処理と発話計画に関与しており, 統語変換と二重課題要求(SOVからSVOへの再順序など)が前頭脳領域を活性化することを再び示している.

これらの研究を総合すると, fMRI/PET研究は同時通訳が言語領域(理解/産出用)と実行制御領域(注意と課題管理用)のネットワークに依存するという考えに収束している.
重要なことに, これらの制御ネットワークは, Seeberのモデルが予測するような構造的非対称性処理時により一層活性化されると推測される.

要約すると, 同時通訳は前頭実行および基底核回路を集中的に活性化し, 理解, 翻訳, 産出を並行して処理するために極度の認知制御が必要であることを反映している.
さらに, 長期的な通訳経験は認知制御ネットワークの構造的・機能的再組織化をもたらし, 高認知負荷条件下での処理効率を向上させる.
これらの神経科学的知見は, 言語間構造的非対称性が同時通訳の認知負荷に与える影響についてのSeeberのCLMの理論的枠組みを神経レベルで支持するものである.

\subsection{machine interpretation system approaches toward simultaneous interpretation tasks}

同時機械翻訳の分野では, 高品質な出力を維持しながら低遅延を実現するという根本的な課題に対して, 様々なアプローチが提案されている.
この課題の核心は「いつ部分翻訳を出力するか」を決定する決定ポリシー(decision policy)にある.
同時翻訳システムは入力を受け取りながらリアルタイムで翻訳を出力する際, 各時点で追加入力を待つ(READ)か部分翻訳を出力する(WRITE)かを選択する必要がある.

決定ポリシーには固定ポリシーと適応的ポリシーの2つのアプローチが存在する.
固定ポリシーは事前に定められたルール(例:k個の単語を待ってから翻訳開始)に従って動作するが, 入力内容に関係なく一定パターンで処理するため効率性に限界がある.
一方, 適応的ポリシーは現在の入力内容(文章の複雑さ, 語順の違い, 文脈の明確さ等)に基づいて動的に決定を行うため, より効率的な翻訳が可能となる.

しかし従来の適応的ポリシーは専用の複雑なアーキテクチャを必要とし, 高い計算コストを伴う問題があった.
また, 多くの手法では同時条件をシミュレートするため, 訓練時に部分入力を提供する必要があり, これが訓練と推論条件の不一致による品質低下を招いていた.
このような背景から, オフライン学習済みモデルを直接同時翻訳に適用できる効果的な適応的ポリシーの開発が求められていた.

Papi et al. (2023) \cite{papi2023attention}は同時音声翻訳における適応的決定ポリシーとしてEDATT (Encoder-Decoder Attention)を提案した.
この手法はオフライン学習済みモデルのencoder-decoder注意パターンを活用して出力タイミングを決定する革新的なアプローチである.
具体的には, システムの注意が音声入力の最新フレーム(直近の音声)に集中している場合, まだ情報が不十分と判断して追加入力を待つ(READ).
一方, 注意が過去のフレーム(古い音声)に分散している場合, 十分な情報が得られたと判断して部分翻訳を出力する(WRITE).
この仕組みにより, システムは入力内容に応じて最適なタイミングで翻訳を出力できる.
英語→ドイツ語・スペイン語の実験において, 従来手法と比較してBLEUスコアで最大7点の改善を達成した.
この研究は注意機構を利用した適応的同時翻訳の有効性を実証している.

現行の同時機械翻訳システムの多くは, カスケードモデル(cascade model)と呼ばれるアーキテクチャを採用している.
カスケードモデルは3つの独立した要素技術を順次組み合わせて実現される:(1)自動音声認識(ASR)システムによる音声からテキストへの変換, (2)機械翻訳(MT)システムによる原言語から目標言語への翻訳, (3)音声合成(TTS)システムによるテキストから音声への変換.
このアプローチの利点は, 各要素技術が独立して開発・改良できることにある.
しかし, 各段階でのエラーが累積的に伝播し, 全体の性能を劣化させる問題がある.
さらに, 音声認識の出力は連続的なテキストストリームであるため, 機械翻訳システムに送る前に適切な文単位に分割する必要がある.

一方, ストリーミング機械翻訳においては文分割処理が別の重要課題となっている.
従来のカスケード型アプローチでは音声認識システムの出力を文単位に分割してから翻訳システムに送る必要があり, この中間的な分割ステップがシステム全体のボトルネックとなっていた.
特に, 機械翻訳システムは通常数百トークンの文で訓練されているが, 実際のライブセッションでは数千トークンに及ぶストリームを処理する必要があり, 訓練と推論の条件に大きなミスマッチが生じていた.
さらに, 独立した分割モデルの品質に翻訳品質が大きく依存するため, 分割エラーが翻訳品質に直接的な悪影響を与えるという問題もあった.

Iranzo-Sánchez et al. (2023) \cite{iranzo2023segmentation}はストリーミング機械翻訳における分割処理の問題に着目し, Segmentation-Freeフレームワークを提案した.
従来手法では文分割の決定を翻訳前に行う必要があったが, 提案手法ではこの決定を翻訳後に遅延させる革新的なアプローチを採用した.
具体的には, 翻訳モデルが未分割の入力ストリームを直接処理し, 翻訳生成と同時に特別トークン「[SEP]」を用いて適切な分割位置を決定する.
メモリ機構により既翻訳部分と未翻訳部分を管理し, 翻訳品質に影響する分割エラーを削減できる.
英独・英仏・英西の実験において, 従来の分割ベース手法を品質・レイテンシの両面で上回る結果を示した.

さらに発展的な研究としては音声の処理から、その後のアウトプット出力の音声の産出までをエンドツーエンド、つまり一貫して行うアプローチの研究が進んでいる。
これは、Source Languageを音声として受け取り、最終的な出力を音声で行うというものである。
しかし、この研究も現状ではまだ実用化されている事例はなく、これはモデル自体が非常に大きい処理リソースを要求するためだと推測される。

土肥ら (2024) \cite{doi2024evaluation}は英日同時機械翻訳システムの評価において順送り訳データの有効性を検証した.
従来のオフライン翻訳データを参照訳とする評価では, 流暢さを優先して原発話と語順が大きく異なる訳出が高く評価される傾向がある.
一方で同時通訳データを参照訳とする評価では, 通訳者の省略や要約により原発話の内容が欠落するため, モデル性能を過小評価してしまう問題がある.
順送り訳データは原発話の内容を保持しながら同時通訳らしい語順を維持するため, 同時機械翻訳モデルの評価により適している.
実験結果では順送り訳データを参照訳として評価した場合, 同時通訳データで学習したモデルが最高のBLEUスコア(15.982)を達成し, 語順を考慮することの重要性を実証した.