\section{Background and Related Work}

\subsection{Cognitive Models of Simultaneous Interpretation}

Gileの努力モデル(Effort Model; EM)は同時通訳における認知的プロセスを理解する上で基礎的な枠組みを提供したが, いくつかの限界があった.
Kahneman[引用するよ~]の単一資源理論に基づくこのモデルは, 全ての認知タスクが1つの未分化な資源プールを競合すると仮定している.
しかし, この理論では完璧な時分割(perfect time-sharing)現象を説明できず, タスク構造の変化が異なる干渉度を生み出すことも説明できない.
[引用するよ~]

Seeberの認知負荷モデル(Cognitive Load Model; CLM)はGileのモデルの限界を克服するためWickens[引用するよ~]の多重資源理論に基づいて開発された.
このモデルは同時通訳を言語理解タスクと言語生成タスクのリアルタイムな組み合わせとして捉え, 構造的に類似したタスク間でより強い干渉が生じると予測する.
また, 入力と出力の両方の特徴を考慮した需要ベクトル(demand vectors)を用いて, 局所的な認知負荷を詳細に分析することができる.
[引用するよ~]

言語間の構造的非対称性に対処するため, Seeberは通訳者が用いる4つの認知戦略を特定した.
第一の戦略である待機(waiting)は, より多くの原言語情報を得るために目標言語の産出を一時停止する戦略である.
この戦略により通訳者は認知負荷を一時的に軽減できるが, 情報をワーキングメモリに保持する必要があり, 下流での認知負荷の大幅な増加を招く可能性がある.
[引用するよ~]

第二の戦略である時間稼ぎ(stalling)は待機と同様に時間を稼ぐことを目的とするが, 沈黙の代わりに"中性的な埋め草"を産出する.
この戦略は聞き手や通訳者自身の不快感を軽減するが, 埋め草の符号化と産出が理解プロセスと重複するため処理の複雑さを増す.
また, 待機戦略と同様に通訳者の遅延時間(lag)を蓄積し, 全体的な認知負荷の増加をもたらす.
[引用するよ~]

第三の戦略であるチャンキング(chunking)は, 文を完全に展開されるのを待たずに符号化できる小さな断片に入力を分割する戦略である.
この戦略では原言語入力を即座に統合・符号化できるが, 引数間を関連付ける主動詞の不在により断片を下流で繋ぎ合わせる必要が生じる.
結果として時間的に遅延した認知負荷の増加と, しばしば不自然で"言語に暴力を加える"ような構文を生成する可能性がある.
[引用するよ~]

第四の戦略である予測(anticipation)は, 話者による発話に先立って原談話の一部を予測する能力である.
この戦略は推論処理("推測")に伴う認知資源を除いて, 認知負荷をベースライン値に近く維持することができる.
また, ベースライン値に近い遅延時間で通訳を完了でき, 他の戦略で見られるスピルオーバー効果を回避できる理想的な解決策である.
しかし, 予期しない動詞による"驚き"のリスクが伴い, 文の重要な意味的・文体的要素を損なう危険性がある.
[引用するよ~]

Seeberの研究では認知負荷の定量化を実現するため, 瞳孔反応測定法(pupillometry)という心理生理学的手法を採用した.
この手法は認知活動に伴う瞳孔径の変化を測定することで, 意識的制御が困難な客観的な認知負荷指標を提供する.
Wickensの多重資源理論に基づいた干渉係数(conflict coefficients)と組み合わせることで, 各通訳戦略における局所的認知負荷の変化を数値的に表現することを試みている.
[引用するよ~]

\subsection{Affect of Grammatical Structure on Cognitive Load during SI by professional interpreters}

Seeber \& Kerzel (2012) \cite{seeber2012cognitive}は瞳孔測定法(pupillometry)を用いてCLMの予測を実証的に検証した.
ドイツ語の動詞末構造から英語への同時通訳実験において, 10名のプロ通訳者を対象とした心理生理学的測定を実施した.
実験では動詞初期構造(verb-initial)をベースラインとし, 動詞末構造(verb-final)との認知負荷を比較した.
瞳孔径の変化を250Hzで連続測定することで, 同時通訳中の局所的認知負荷をリアルタイムで定量化することに成功した.

実験結果では, 動詞末構造の通訳時に文末付近(Period of Interest 4)で瞳孔径が有意に拡大し, 認知負荷の顕著な増加が確認された.
この負荷増加は予測された時点で現れ, CLMが示す「下流への負荷輸出(exported load)」現象を実証した.
また文脈あり条件では文脈なし条件と比較して認知負荷が軽減される傾向が観察され, 推論処理による負荷軽減効果が示唆された.
重要な点として, 実験中に認知的過負荷(cognitive overload)を示す急激な瞳孔収縮は観察されず, 通訳者が能力限界内で作業していることが確認された.
これらの結果は構造的非対称性が同時通訳に実質的な認知的コストを課すことを客観的に立証した.

Seeber (2013) \cite{seeber2013cognitive}は認知負荷測定手法の包括的な分析を行い, 瞳孔測定法の同時通訳研究における有効性を論じた.
心理生理学的手法としての瞳孔測定法は, 主観的方法や分析的方法と比較して客観性と時間分解能の点で優位性を持つ.
認知活動に伴う瞳孔拡大は刺激提示後300-500msで開始し, 交感神経系の自動的反応として意識的制御が困難である.
この手法により句レベルや文レベルでの局所的認知負荷変化を捉えることができ, 同時通訳の動的な認知プロセス解明に貢献している.



\subsection{Affect of Grammatical Structure on Peformance of machine interpretation systems}

Papi et al. (2023) \cite{papi2023attention}は同時音声翻訳における適応的決定ポリシーとしてEDATT (Encoder-Decoder Attention)を提案した.
この手法はオフライン学習済みモデルのencoder-decoder注意パターンを活用して出力タイミングを決定する.
音声入力の最新フレームに注意が集中している場合は追加の入力を待ち, そうでない場合は部分仮説を出力するという戦略を採用している.
英語→ドイツ語・スペイン語の実験において, 従来手法と比較してBLEUスコアで最大7点の改善を達成した.
この研究は注意機構を利用した適応的同時翻訳の有効性を示している.

Iranzo-Sánchez et al. (2023) \cite{iranzo2023segmentation}はストリーミング機械翻訳における分割処理の問題に着目し, Segmentation-Freeフレームワークを提案した.
従来のカスケード型アプローチでは前処理として文分割が必要であり, この分割エラーが翻訳品質に悪影響を与えていた.
提案手法では翻訳モデルが未分割の入力ストリームを直接処理し, 翻訳生成後に分割決定を行う.
メモリ機構により既翻訳部分と未翻訳部分を管理し, 特別トークン「[SEP]」を用いて分割を表現する.
英独・英仏・英西の実験において, 従来の分割ベース手法を品質・レイテンシの両面で上回る結果を示した.

土肥ら (2024) \cite{doi2024evaluation}は英日同時機械翻訳システムの評価において順送り訳データの有効性を検証した.
従来のオフライン翻訳データを参照訳とする評価では, 流暢さを優先して原発話と語順が大きく異なる訳出が高く評価される傾向がある.
一方で同時通訳データを参照訳とする評価では, 通訳者の省略や要約により原発話の内容が欠落するため, モデル性能を過小評価してしまう問題がある.
順送り訳データは原発話の内容を保持しながら同時通訳らしい語順を維持するため, 同時機械翻訳モデルの評価により適している.
実験結果では順送り訳データを参照訳として評価した場合, 同時通訳データで学習したモデルが最高のBLEUスコア(15.982)を達成し, 語順を考慮することの重要性を実証した.



\subsection{Neural Correlates of Cognitive Load during Structural Asymmetry Processing}

Hervais-Adelman et al. (2015) \cite{hervais2015fmri}は同時通訳の神経基盤を解明するため, 50名の多言語話者を対象としたfMRI研究を実施した.
同時通訳時には音声知覚・産出領域に加えて, 尾状核や被殻を含む基底核ネットワークが特異的に活性化することが示された.
特に尾状核は語彙意味選択の制御に, 被殻は発話出力の制御に関与し, 複数言語制御における実行機能ネットワークの重要性が明らかになった.
これらの知見は同時通訳がバイリンガルの通常の言語切り替えを超える極度の言語制御を要することを神経学的に実証している.

Lin et al. (2018) \cite{lin2018costly}は中国語から英語への同時通訳において語順変換処理の神経基盤を検討した.
fNIRS(functional near-infrared spectroscopy)を用いた計測では, 語順を再構成する「パラフレーズ」戦略時に左前頭前野の活動が広範囲に増加した.
特にブローカ野を含む下前頭回において即座かつ強力な活動が観察され, SOV→SVO変換時の文法処理負荷を反映している.
この結果はSeeberの認知負荷モデルにおける構造的非対称性処理の神経相関を直接的に裏付けるものである.

通訳訓練による脳の適応的変化も報告されている.
Hervais-Adelman et al. (2015) \cite{hervais2015plasticity}は1年間の通訳訓練前後のfMRI計測により, 訓練後に右尾状核の活動が有意に低減することを発見した.
この変化は熟達に伴う多言語制御の自動化を示唆し, 認知負荷軽減の神経機構を明らかにしている.
また訓練群と対照群では脳活動パターンの変化が明確に異なり, 通訳訓練による特異的な脳機能再編が確認された.

参考文献:
	•	Alexis Hervais-Adelman et al. (2015). “fMRI of Simultaneous Interpretation Reveals the Neural Basis of Extreme Language Control.” Cerebral Cortex, 25(12), 4727-4739  .
	•	Alexis Hervais-Adelman et al. (2015). “Brain functional plasticity associated with the emergence of expertise in extreme language control.” NeuroImage, 114, 264-274  .
	•	Stefan Elmer et al. (2014). “Processing demands upon cognitive, linguistic, and articulatory functions promote grey matter plasticity in the adult multilingual brain: Insights from simultaneous interpreters.” Cortex, 54, 179-189  .
	•	Eowyn Van de Putte et al. (2018). “Anatomical and functional changes in the brain after simultaneous interpreting training: A longitudinal study.” Cortex, 99, 243-257  .
	•	Xiaohong Lin et al. (2018). “Which is more costly in Chinese to English simultaneous interpreting, ‘pairing’ or ‘transphrasing’? Evidence from an fNIRS neuroimaging study.” Neurophotonics, 5(2), 025010  .
	•	Haruko Yagura et al. (2021). “Selective Attention Measurement of Experienced Simultaneous Interpreters Using EEG Phase-Locked Response.” Frontiers in Human Neuroscience, 15:581525  .
	•	Carina Klein et al. (2018). “The interpreter’s brain during rest — Hyperconnectivity in the frontal lobe.” PLOS ONE, 13(8): e0202600 .