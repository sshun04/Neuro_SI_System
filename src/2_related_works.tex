\section{Background and Related Work}

\subsection{Cognitive Models of Simultaneous Interpretation}

Daniel Gileの努力モデルは、同時通訳を聞き取り(L)・産出(P)・記憶(M)・調整(C)の4つの認知努力の組み合わせとして定式化した。Seeberはこれを発展させ、Wickensの多重リソース理論に基づいて認知負荷を定量化し、非対称言語構造への対処として4つの戦略を特定した。最新の神経画像研究(2020-2024年)により、これらの認知戦略が特定の脳領域と対応することが明らかになっている\cite{ishizuka2024two}。

\subsection{Computational Neuroscience Approaches}

計算論的神経科学の観点から、同時通訳の認知プロセスをモデル化する試みが行われている\cite{dangelo2013realistic}。これらの研究は、現実的なニューロンとネットワークのモデリングを通じて、脳のシミュレーションを目指している\cite{mcdougal2016reproducibility}。
