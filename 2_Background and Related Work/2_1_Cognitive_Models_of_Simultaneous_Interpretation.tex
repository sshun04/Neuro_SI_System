\subsection{Cognitive Models of Simultaneous Interpretation}

Daniel Gileの努力モデルは、同時通訳を聞き取り(L)・産出(P)・記憶(M)・調整(C)の4つの認知努力の組み合わせとして定式化した。Seeberはこれを発展させ、Wickensの多重リソース理論に基づいて認知負荷を定量化し、非対称言語構造への対処として4つの戦略を特定した。最新の神経画像研究(2020-2024年)により、これらの認知戦略が特定の脳領域と対応することが明らかになっている。

同時通訳は前頭前野、基底核、側頭葉、頭頂葉を中心とする広域神経ネットワークを動員し、熟練通訳者では「より効率的で集約的な神経活動パターン」を示すことが実証されている。Seeberの認知負荷モデルの各構成要素(P-C-R-S)は特定の脳領域と対応し、SOV-SVO語順変換では特に「前頭前野と基底核の活動増加」が認められることが判明した。 