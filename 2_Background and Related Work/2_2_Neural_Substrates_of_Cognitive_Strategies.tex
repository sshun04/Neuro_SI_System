\subsection{Neural Substrates of Cognitive Strategies}

各認知戦略は異なる神経基盤を持つ。\textbf{待機(Waiting)戦略}は右下前頭回を中心とする抑制制御ネットワークに依存し、前帯状皮質が競合モニタリングシステムとして機能する。\textbf{引き延ばし(Stalling)戦略}では、基底核が時間制御を通じて中心的役割を果たし、尾状核が語彙意味選択と制御に、被殻が継続的な言語出力制御を担う。

\textbf{チャンク化(Chunking)戦略}は分散作業記憶ネットワークを動員し、前頭前野-基底核回路の相互作用により適応的チャンク化が作業記憶容量を改善する。\textbf{予測(Anticipation)戦略}は階層的予測ネットワークを含み、後部領域(視覚皮質)は1-4秒先を、前部領域(前頭前野)は8-15秒先を予測する時間的階層性を示す。 