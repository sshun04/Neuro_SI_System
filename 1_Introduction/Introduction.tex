# Introduction
大規模言語モデル(LLM)が多くの認知タスクで人間を上回る性能を示す中、同時通訳は依然として熟練した人間通訳者がAIシステムを上回る効率性と精度を維持している分野である。現在のAI同時通訳システムは、音声認識・機械翻訳・音声合成の組み合わせによって実現されているが、膨大な計算リソースを必要とし、リアルタイム性に課題を抱えている。

本研究では、Seeberの認知負荷モデルで提唱された4つの認知戦略(Waiting, Stalling, Chunking, Anticipation)をAI同時通訳システムに適用し、計算効率の大幅な改善が可能かを検証する。これらの戦略は、ドイツ語(SOV)から英語(SVO)への通訳における語順変換の課題に対処するために開発された。本研究の目的は、2020-2024年の神経画像研究で明らかになった神経基盤に基づく認知戦略をAIアーキテクチャに組み込むことで、パラメータ効率と処理速度の両面で改善を実現することである。