\subsection{Limitations and Neurobiological Constraints}

現在の実装は神経活動の主要パターンに焦点を当てており、ミリ秒精度の神経動態や複雑な神経化学的相互作用は簡略化されている。また、個人差の大きい神経可塑性パターンの完全な再現には、より詳細な神経プロファイリングが必要である。マルチモーダル要素(韻律、話者識別、視覚的文脈)の統合では、対応する神経基盤(聴覚皮質、顔認識領域、視覚皮質)の複雑な相互作用を考慮する必要がある。

神経保護効果や認知的老化予防といった長期的な神経可塑性効果は、現在のAIシステムでは直接的に模倣できない生物学的特性であり、今後の研究課題として残されている。 