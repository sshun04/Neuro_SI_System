\subsection{Neural Basis Validation of Cognitive Strategies}

神経科学的知見をAIアーキテクチャに組み込むことで、単なる性能向上を超えた生物学的妥当性を持つシステムを実現できることを実証した。特に、熟練通訳者の神経効率性パターンである「より少ない神経活動でより良いパフォーマンス」を6.2倍のパラメータ効率改善として再現したことは、神経科学的知見の工学的価値を示している。

Seeberの認知負荷モデルの神経基盤(P成分:上側頭回、C成分:広範言語ネットワーク、R成分:運動系、S成分:作業記憶ネットワーク)を計算アーキテクチャに直接対応付けることで、従来の純粋データ駆動型とは異なる原理に基づくAI設計が可能となった。 