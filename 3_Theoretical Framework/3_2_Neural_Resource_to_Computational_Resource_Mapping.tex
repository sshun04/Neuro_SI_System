\subsection{Neural Resource to Computational Resource Mapping}

人間の神経リソースとAIの計算リソースの対応関係を神経科学的知見に基づいて確立した。Korenar et al. (2023)の研究により明らかになった尾状核・被殻の「拡大-再正常化」パターンを参考に、初期学習段階ではバッファサイズを拡大し、熟練段階では効率化による正常化を実装する。ワーキングメモリ容量は背外側前頭前野の活動に対応し、処理能力は前頭シータパワー(4-8Hz)の変調パターンに基づいて計算複雑度として実装される。

熟練通訳者が示す神経効率性パターン、すなわち「優れたパフォーマンスを維持しながら主要脳領域でより少ない活動を示す」特性を、パラメータ効率性の最適化目標として設定する。この対応関係により、認知戦略を神経科学的妥当性を持つアルゴリズムとして実装可能となる。 