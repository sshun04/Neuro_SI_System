\subsection{Neural-Inspired System Architecture}

神経基盤に基づくCognitiveInterpretationSystemを実装した。\textbf{Waiting戦略}では、Millerの7±2制約と右下前頭回の抑制制御機能を参考に、バッファ管理システムを構築した。動詞検出時には、ブローカ野(BA 44)の統語的再配列機能を模倣した構造化出力生成を行う。

\textbf{Anticipation戦略}では、階層的予測ネットワークの時間的階層性を実装し、短期予測(1-4秒)と長期予測(8-15秒)を組み合わせた予測システムを構築した。Hervais-Adelman et al. (2015)の知見に基づき、前頭・側頭言語領域を含むネットワーク活性化パターンを模倣し、産出段階でのブローカ野活動増強を再現する。 