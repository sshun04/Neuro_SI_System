\section{Conclusion}

本研究では、2020-2024年の神経画像研究で明らかになった同時通訳の神経基盤に基づき、人間の認知戦略をAIシステムに適用することで、従来システムの6.2倍のパラメータ効率を実現できることを実証した。特に、熟練通訳者の神経効率性パターンを模倣したAnticipation戦略は、階層的予測ネットワークの実装により計算要求を大幅に削減しながら同等の性能を維持した。

この成果は、神経科学的知見をAI設計に活用するバイオインスパイアードAIの有効性を実証するものであり、単なる工学的最適化を超えた生物学的妥当性を持つAI開発の新しいパラダイムを提示している。Seeberの認知負荷モデルの神経基盤を計算アーキテクチャに直接対応付けることで、効率的で理論的に根拠のあるAIシステム設計が可能となった。

今後は、より詳細な神経動態の実装、個人差対応の神経可塑性モデル、マルチモーダル神経ネットワークの統合を通じて、認知志向AI設計の可能性を完全に実現する必要がある。この研究は、複雑な言語処理タスクにおける人間の神経基盤とAI計算の対応関係の探求基盤を確立した。 