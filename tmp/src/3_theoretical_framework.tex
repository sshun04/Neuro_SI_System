\section{Theoretical Framework}

前章では, 同時通訳における認知負荷と神経基盤に関する既存研究を概観し, この分野の研究動向と課題を明らかにした.
それを受けて本章では, 本研究の理論的基盤となるシーバーの認知負荷モデルと同時通訳に関わる神経基盤について詳述し, 両者の関連性を検討する.

同時通訳 (SI) は, 極めて高い認知負荷と言語制御の要求を伴うタスクであり, その神経基盤に関する研究が進められている.
シーバーの認知負荷モデル (CLM) は, 同時通訳における「理解」「産出」「ワーキングメモリにおける情報保持」という重なり合う構成要素のタスクによって生成される瞬間のメモリ負荷を図式化する.
特に, ドイツ語の動詞最終構文 (SOV) を英語のような動詞先頭言語 (SVO) に通訳する際の認知負荷の増加を予測しており, 文末 (PI 4) での負荷が最も顕著であるとされる.
このモデルは, 認知負荷を管理するための戦略として予測 (anticipation) や引き延ばし (stalling) などを提示している.

fMRI や PET, EEG を用いた研究により, 同時通訳者の脳では, 高度な言語処理と実行機能に対応するための構造的・機能的な適応が見られることが明らかになっている.
主要な脳領域として, 尾状核 (Caudate Nucleus) はマルチリンガル言語制御, 学習, 運動制御, およびより広範な実行機能に関与する.
特に, 言語セットの選択と制御という高次モニタリングの役割を担い, 状況に応じた行動の決定に関わると考えられている.
訓練後には活動が減少することが示されており, これはタスクの自動化が進むことで, マルチリンガル言語制御への要求が低下するためと考えられる.
また, SI の実践量が多いほど, 両側の尾状核の灰白質体積が減少するという負の相関が報告されている.

被殻 (Putamen) は瞬間的な言語出力制御に関与し, 不適切な言語での発話の抑制や発話の調音の調整を助けると考えられている.
聞き取りと発話の重複期間 (同時性) によって被殻の活動が変調されることが示されており, これは同時に複数の処理を行う際の瞬間的な制御負荷を反映している.
前帯状皮質 (Anterior Cingulate Cortex: ACC) / 背側前帯状皮質 (dACC) は葛藤モニタリング (conflict monitoring), 認知制御, バイリンガル言語制御に深く関与する.
バイリンガルの脳における葛藤モニタリングの適応は, ACC における活動の増加として観察される.

補足運動野 (Supplementary Motor Area: SMA) / 前補足運動野 (pre-SMA) は発話の開始, 言語の切り替え, 運動制御, バイリンガル言語制御に関与する.
特に, 言語切り替え時の発話開始において重要な役割を担うとされている.
下前頭回 (Inferior Frontal Gyrus: IFG) / ブローカ野 (Broca's Area) は意味処理, 構音の計画と実行, 言語制御, 言語切り替え, ドメイン一般の認知制御に関わる.
同時通訳では, 文法的に複雑な構造の統合や構音の計画と実行に特に関連すると考えられ, ブローカ野の灰白質体積が通訳訓練の累積時間と相関することが報告されている.

上側頭回 (Superior Temporal Gyrus: STG) / 聴覚関連皮質 (Auditory-Related Cortex: ARC) / 側頭平面 (Planum Temporale) は聴覚知覚, 言語理解, 音と構音の関連付け, 音素処理に関与する.
音声から意味へのマッピング (腹側経路) と音声から構音へのマッピング (背側経路) のインターフェースとして機能する.
熟練した同時通訳者では, 特にこの左側の背側経路における機能的結合性 (シータ帯域での同期) が強化されることが示されている.
小脳 (Cerebellum) は複雑な発話行動の自動化, 運動機能の調整, 言語制御ネットワークの一部として関与し, SI 中の複雑な発話行動の自動化に寄与すると考えられている.

頭頂葉 (Parietal Lobes) / 下頭頂小葉 (IPL) / 角回 (Angular Gyrus) / 縁上回 (Supramarginal Gyrus) はタスク表現の維持, 注意制御, ワーキングメモリ機能に関与し, バイリンガルにおいて構造的変化が見られる.
角回は, 言語制御だけでなく, 超モダールな注意制御や意味制御にも関連付けられている.
島 (Insula) は発話の調音や発話開始, 運動感覚変換に関与する.

シーバーの認知負荷モデルの予測と神経基盤の発見は密接に対応している.
CLM が予測する文末での高い認知負荷 (特に PI 4) は, ワーキングメモリと実行制御に大きな負担をかける.
この負荷を処理するために, 尾状核, 被殻, ACC, SMA, IFG/ブローカ野, 頭頂葉など, 実行機能や言語制御, ワーキングメモリに関わる脳領域が, より活発に活動したり, その結合が強化されたりすると考えられる.
特に, 予測 (anticipation) 戦略は, ラグを短縮する一方で, 動詞を「推測する」という推論処理自体が認知資源を必要とするため, 局所的な認知負荷が増加するとされる.
この「予測」は, 尾状核が担う機能の一つであるという神経科学的知見と符合する.

シーバーのモデルでは, 「引き延ばし (stalling) 」戦略が通訳者のラグを大幅に増加させ, 意味のない言葉で間を埋める (padding) ことが理解プロセスと重なるため, 処理の複雑さ (および認知負荷) が増すことが示されている.
このようなマルチタスク処理は, 言語生成とモニタリングに関わる被殻や SMA, ブローカ野などの活動増加として現れる可能性がある.
CLM は, 談話文脈 (discourse context) がある場合, 単一文脈 (sentence context) に比べて, 同時通訳中の認知負荷が軽減される傾向があることを示唆している.
これは, 角回のような脳領域のより効率的な関与によって媒介される可能性があり, 角回は超モダールな注意制御と意味制御に関連付けられている.

シーバーのモデルが認知負荷を管理するための戦略 (例: 予測) を示唆するように, 神経科学的研究は, 通訳者が訓練によって可塑的変化を遂げ, タスクが自動化されるにつれて一部の領域 (例: 右尾状核) の活動が減少することを示している.
これは, 熟練した通訳者が, たとえ認知負荷が高い状況であっても, より効率的に対処できるという考えと一致する.
同時通訳訓練による脳の構造的結合性の増加 (例えば, 前頭領域や基底核における結合性の増加) は, 極端な言語制御に対応するための脳の適応的な再構築を裏付けている.
特に, 左側の音と構音の関連付け (ARC からブローカ野) における機能的結合性 (シータ帯域での同期) の増加は, 通訳者が翻訳のために発話コードを事前に活性化するという, 予測戦略の重要な側面を裏付けるものである.

現在の機械同時通訳システムと人間の同時通訳には, 根本的なアプローチの違いが存在する.
この違いを理解することは, より効果的な同時通訳システムを構築するために不可欠である.
以下では, 機械システムと人間の処理方式の違いを具体的に検討し, その課題と今後の改善方向を明らかにする.

現在の機械同時通訳システムは, 基本的に逐次処理 (sequential processing) パラダイムに基づいている.
例えば, 「私は昨日東京で友人と会いました」という文を英語に通訳する場合, 機械システムは「私は (I)」→「昨日 (yesterday)」→「東京で (in Tokyo)」→「友人と (with a friend)」→「会いました (met)」という順序で, 各要素を順番に処理し出力する.
このアプローチでは, 時間軸に沿って単一の処理ストリームが進行し, 各時点では一つの言語処理タスクのみが実行される.
ストリーミング音声認識技術を用いたシステムでも, 音声の到着順序に従って翻訳処理が実行され, 基本的には一方向的な情報の流れに依存している.

より高度なアテンション機構を持つニューラル機械翻訳システムであっても, この逐次性の制約は変わらない.
アテンション機構は確かに翻訳タイミングを動的に決定できるが, それでも「入力の理解→翻訳の生成→出力の産出」という基本的な時系列処理の枠組みを超えることはない.
例えば, ドイツ語の動詞後置文「私は信じる, 代表団が彼らの決定を長い議論の後で下す (Ich glaube, dass die Delegierten ihre Entscheidung nach einer langen Debatte treffen)」では, 動詞「treffen (下す)」が文末に現れるまで, システムは完全な翻訳を開始できない.
このため, 機械システムは文末まで待機するか, 不完全な情報に基づいて推測的翻訳を行う必要がある.

一方, 人間の同時通訳者は, 本質的に並列処理 (parallel processing) を行っている.
シーバーの認知負荷モデルが示すように, 熟練した通訳者は「聴取・理解」「ワーキングメモリでの保持」「目標言語への変換」「発話の産出」「エラーの監視」という複数のタスクを同時並行で実行する.
例えば, 上述のドイツ語の例では, 通訳者は「代表団が (the delegates)」を聞いている最中に, 既に「私は信じる (I believe)」を英語で発話し始めることができる.
同時に, ワーキングメモリでは「長い議論の後で (after a long debate)」を保持しながら, 文脈から動詞の内容を予測 (anticipation) することも可能である.

この並列処理能力は, 前述した神経基盤の特殊化によって支えられている.
尾状核による予測機能, 被殻による瞬間的言語制御, ブローカ野による構文処理, 上側頭回による音韻処理などが, 相互に協調しながら同時に活動する.
特に重要なのは, 左側の背側経路における音と構音の関連付け機能であり, これにより通訳者は聴取中の音韻を即座に発話準備に変換できる.
このような神経ネットワークの並列性は, 機械システムの逐次処理では模倣が困難である.

さらに, 人間の通訳者は文脈依存的な推論と修正能力を持つ.
談話文脈 (discourse context) が利用可能な場合, 通訳者は局所的な認知負荷を軽減し, より効率的な処理を行うことができる.
例えば, 政治演説の文脈であれば, 「決定 (decision)」という語が出現した時点で, 通訳者は「政策決定 (policy decision)」や「重要な決定 (important decision)」といった適切な修飾語を予測し, 準備することが可能である.
このような文脈的推論は, ワーキングメモリの負荷を軽減し, より自然で流暢な通訳を可能にする.

現在の機械同時通訳システムの限界は, この並列処理能力の欠如にある.
逐次処理パラダイムでは, ある処理が完了するまで次の処理を開始できないため, 人間のような柔軟で効率的な通訳は実現困難である.
特に, SOV (主語-目的語-動詞) 構造の言語から SVO (主語-動詞-目的語) 構造の言語への通訳では, この制約が深刻な遅延と品質低下を引き起こす.
また, 大規模言語モデルの巨大化傾向は, リアルタイム処理に必要な計算資源の増大を招き, 同時通訳の即時性要件との矛盾を生じさせている.

これらの課題を解決するためには, 人間の神経処理に学んだ新しいアーキテクチャが必要である.
具体的には, 複数の処理ユニットが並列に動作し, 相互に情報を交換しながら協調的に通訳を生成するシステムが求められる.
このようなシステムでは, 音韻処理, 意味理解, 構文変換, 発話生成が独立したモジュールとして並列動作し, 認知負荷の動的な分散と最適化が可能になる.
本研究では, この理論的基盤に基づいて, 人間の認知処理を模倣した並列型同時通訳システムの開発を進める.

