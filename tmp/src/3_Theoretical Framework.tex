\section{Theoretical Framework}

\subsection{Cognitive Strategy Formalization with Neural Mapping}

4つの認知戦略を神経基盤に基づいて計算論的に定式化した。\textbf{Waiting戦略}では、右下前頭回の抑制制御機能を模倣し、\texttt{CognitiveLoad\_waiting(t) = α·BufferSize(t) + β·InhibitionCost(t)}として、バッファサイズと抑制コストの関数で表現する。この戦略では、背外側前頭前野(DLPFC)がワーキングメモリ管理を担い、動詞出現まで情報を保持する。

\textbf{Anticipation戦略}は、内側前頭前野による高レベル物語スキーマ追跡と下前頭回の構造化シーケンス処理を基盤とし、\texttt{CognitiveLoad\_anticipation(t) = γ·PredictionCost(context) + δ·ConfidencePenalty + ε·HierarchicalProcessing}として定式化される。Elmer \& Kühnis (2016)が発見した聴覚皮質(BA 41/42)とブローカ野(BA 44/45)間のシータ帯域(4-7Hz)における強化された機能的結合が、この戦略の神経基盤となる。 


\subsection{Neural Resource to Computational Resource Mapping}

人間の神経リソースとAIの計算リソースの対応関係を神経科学的知見に基づいて確立した。Korenar et al. (2023)の研究により明らかになった尾状核・被殻の「拡大-再正常化」パターンを参考に、初期学習段階ではバッファサイズを拡大し、熟練段階では効率化による正常化を実装する。ワーキングメモリ容量は背外側前頭前野の活動に対応し、処理能力は前頭シータパワー(4-8Hz)の変調パターンに基づいて計算複雑度として実装される。

熟練通訳者が示す神経効率性パターン、すなわち「優れたパフォーマンスを維持しながら主要脳領域でより少ない活動を示す」特性を、パラメータ効率性の最適化目標として設定する。この対応関係により、認知戦略を神経科学的妥当性を持つアルゴリズムとして実装可能となる。 

 
TODO

